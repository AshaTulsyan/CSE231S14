\subsection{Pass Algorithm Description}

We want to write a pass which helps us predict, for each function of the input program, what is the bias that any branch in the function is taken during the execution. Similarly to the previous section, we need bitcode instrumentation to accomplish this task. As done previously, we insert independent functions to the input program described in algorithm 3.

These functions expect a function name as the input and increment the found or taken count for that function. This count is eventually displayed when the program terminates. We insert a call instruction to the \code{branchFound} function right above a branch instruction and the \code{branchTaken} function right above the first instruction of the basic block reached when the branch is taken.

\begin{algorithm}[here]
\textbf{branchFound}($fName$) \Begin{
	\If{$Found.containsKey(fName)$}{
		$Found.valueForKey(fName) += 1$ 
	} \Else {
		$Found.insertKeyValuePair(< fName , 1 > )$
	}
}
\textbf{branchTaken}($fName$) \Begin{
	\If{$Taken.containsKey(fName)$}{
		$Taken.valueForKey(fName) += 1$ 
	} \Else {
		$Taken.insertKeyValuePair(< fName , 1 > )$
	}
}
\textbf{print}() \Begin{
	\ForEach{$ fName \in Found.keys() $}{
		$v_f \gets Found.valueForKey(fName)$ \\
		$v_t \gets Taken.valueForKey(fName)$ \\
		$print ( fName + " " + v_f + " " + v_t + " " + \frac{v_t}{v_f})$
	}
}
\caption{where $Found$ and $Taken$ are C++ maps of the form \code{<string,int>} and $fName$ is a string expected to be an functionName from the input}
\end{algorithm}

\subsection{Physical Implementation Description}
This opt pass builds on the dynamic instrumentation methods applied to the previous pass. The additional requirement is run-time analysis of program logic rather than the run-time observation of it performed previously. \\
In addition to all skills of previous passes, analysing logic requires:\\ 
\begin{itemize}
\item The ability to instrument a particular instruction
\item An ability to dynamically consume target code control flow data
\end{itemize}
Our opt pass locates conditional branches using the basic block iterator to evaluate instructions and a simple logic test of a dynamic cast and class member test to locate conditional branches:\\

\begin{frame}[fragile]
%\frametitle{Inserting source code}
\lstset{language=C++,
	commentstyle=\color{green}\ttfamily,
    basicstyle=\ttfamily,
    keywordstyle=\color{magenta}\bfseries,
    showstringspaces=false,
    morekeywords={BasicBlock,BranchInst}                
}
\begin{lstlisting}
for(BasicBlock::iterator BI = BB->begin(),
BE = BB->end(); BI != BE; ++BI){
   if(isa<BranchInst>(&(*BI)) ) {
     BranchInst *CI = 
     dyn_cast<BranchInst>(BI);
     if (CI->isUnconditional())
       continue; 
     //Else conditional branch found...
\end{lstlisting}
\end{frame}

We dynamically consume control flow data by instrumenting the 'taken' conditional branch of interest. In .ll code a taken branch is the 0'th successor instruction of a conditional branch. We consume the $CI$ pointer created previously to instrument the 'taken' target basic block.\\

\begin{frame}[fragile]
\lstset{language=C++,
	commentstyle=\color{green}\ttfamily,
    basicstyle=\ttfamily,
    keywordstyle=\color{magenta}\bfseries,
    showstringspaces=false,
    morekeywords={BasicBlock,iterator}                
}
\begin{lstlisting}
BasicBlock *block = CI->getSuccessor(0);
BasicBlock::iterator takenInsertPt
= block->getFirstInsertionPt();
\end{lstlisting}
\end{frame}
  
%The purpose of this part is to write a dynamic analysis that computes the per function branch bias. Specifically, we had to count the total number of branches and the number of taken branches that appear in each function and compute the ratio: taken branches/ total branches. In order to achieve this functionality, we once again declared two helper functions and used two C++ map structures to store the number of total and taken branches. Both of these functions get a function name as an attribute and utilize it as the key of the map structure. The helper function branchFound gets injected into the benchmark code every time a conditional branch with two successors (taken and non-taken branch) is spotted. Similarly, the branchTaken helper function gets injected into the taken branch. If these functions get called for the first time in a single function

\subsection{Benchmark Analysis}
Let's have a closer look at one of the benchmark applications and specifically at the gcd.cpp. Function gcd() gets called from the main method with two arguments (72, 32), it checks if the second argument is equal to zero and since it's not, the branch is not taken. Instead it recursively calls itself with arguments (32,72\%32=8). Once again though the second argument is not equal to zero and the branch is not taken but the function gets called again with arguments (8, 32\%8=0). This time the second argument is equal to zero and the branch is taken, thus it returns to the main function and the program terminates successfully. To sum up, there was only one function with a branch, and this branch was encountered three times but it was taken only once, so the bias is going to be 1/3 as we can see at the log file of the output.	