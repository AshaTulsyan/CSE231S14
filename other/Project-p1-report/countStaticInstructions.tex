
\textbf{Problem statement: }\textit{Write a pass that counts the number of static instructions in a program.}
\textbf{Problem instance:}
\begin{itemize}
\item output total number of instructions
\item output per-instruction count
\end{itemize}

\subsection{Pass description}

Because the functionality is a static analysis, it is sufficient to run a pass on the compiled code of each benchmark.
We store instruction op codes and their corresponding count in a C++ map structure. Each time an instruction is found in the source code, it is either added to an existing map entry or a new entry is created and initialized to 1.

A high level algorithm description would be:\\

\begin{algorithm}
 \KwIn{$M$,$I$}
 $t \gets 0$\\
 \ForAll{$i \in I$}
 { 
 	\If{$M$.containsKey($i$)}
 	{
 		$M$.valueForKey($i$)+=1\\
 	}
 	\Else
 	{
 		$M$.insertKeyValuePair($<i$,$1>$)\\
 	}
 }
 \ForAll{keyValuePair $\in M$}
 {
 	print("Found "keyValuePair.value() "counts of: "keyValuePair.key())\\
 	$t$ += keyValuePair.value()\\
 }
 print("total instructions: "$t$)
 \caption{Static instruction count algorithm}
\end{algorithm}
where:\\
$I$ is an input program instruction list in LLVM byte code (.bc) format\\
$i$ is an individual instruction within $I$\\
$M$ is a C++ map of the form <string,int>

\subsection{Static pass implementation summary}
Besides a working proficiency in C++, translating the algorithm to LLVM required an understanding of:
\begin{itemize}
\item how to build and run an opt module on a target benchmark program
\item how to output data in human readable format from an opt module
\item how instructions are represented by LLVM
\item how instructions are accessed by opt modules
\end{itemize}

The most challenging of the above was understanding how instructions are represented by LLVM and accessing them from opt modules.\\

Our opt pass is able to iterate through the source module by implementing the \textbf{runOnModule()} virtual function of a \textbf{ModulePass} class.

\begin{frame}[fragile]
%\frametitle{Inserting source code}
\lstset{language=C++,
    basicstyle=\ttfamily,
    keywordstyle=\bfseries,
    showstringspaces=false,
    morekeywords={runOnModule}                
}
\begin{lstlisting}
    virtual bool runOnModule(Module &M)()
\end{lstlisting}
\end{frame}

Accessing instructions was accomplished by iterating through the \textbf{input source} \textbf{Module} using an \textbf{inst\_iterator}. The Module contains all instructions from the compiled benchmark and the inst\_iterator points at individual instructions.\\

\begin{frame}[fragile]
%\frametitle{Inserting source code}
\lstset{language=C++,
                basicstyle=\ttfamily,
                keywordstyle=\color{blue}\ttfamily,
                stringstyle=\color{red}\ttfamily,
                commentstyle=\color{green}\ttfamily,
                morecomment=[l][\color{magenta}]{\#}
}
\begin{lstlisting}
    	for (Module::iterator m = M.begin(), e = M.end() ; e != m ; ++m) {
			for (inst_iterator I = inst_begin(m), E = inst_end(m) ; I != E ; ++I) {...}
\end{lstlisting}
\end{frame}


\subsection{Benchmark analysis}

%\subsection{Learning outcomes}
